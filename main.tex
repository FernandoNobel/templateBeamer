\documentclass[9pt,table,aspectratio=169]{beamer}

\usetheme{metropolis}

\graphicspath{{./img/}}
\titlegraphic{\vspace{65mm}\hfill \includegraphics[height=0.8cm]{img/logos/ai2.png} \quad \includegraphics[height=0.8cm]{img/logos/marca_UPV_principal_color300.png}}

\title{Sistemas automáticos - Práctica 1}
\subtitle{Simulación de sistemas dinámicos con MATLAB}
\author{Fernando Nóbel Santos Navarro (\textit{fersann1@upv.es})}

\date{Octubre 16, 2020}
\institute{Grado en Ingeniería en Tecnologías Industriales 2020-2021}

\usepackage[framed,numbered,autolinebreaks,useliterate]{mcode} % Matlab code highlight

\usepackage{booktabs}

\setlength{\defaultaddspace}{1.5pt}
\setlength{\tabcolsep}{5pt}
\usepackage{tabularx}
\newcolumntype{P}{>{\raggedright\arraybackslash}m}
\newcolumntype{C}{>{\centering\arraybackslash}m}

\usepackage{multirow} % \multirow macro

\begin{document}

\maketitle

\begin{frame}{Table of contents}
  \setbeamertemplate{section in toc}[sections numbered]
  \tableofcontents[hideallsubsections]
\end{frame}

\section{Introducción}

\begin{frame}{Introducción}

  \begin{columns}[T,onlytextwidth]
    \column{0.49\textwidth}
    \metroset{block=fill}
    \begin{exampleblock}{Prácticas 1 y 2}
      \begin{enumerate}
        \item Diseño con Red de Petri.
        \item Uso de botonera para simular entradas.
      \end{enumerate}
    \end{exampleblock}

    \column{0.49\textwidth}
    \metroset{block=fill}
    \begin{alertblock}{Prácticas 3 y 4}
      \begin{enumerate}
        \item Diseño con GRAFCET.
        \item Uso del panel de cilindros neumáticos.
      \end{enumerate}
    \end{alertblock}

  \end{columns}
\end{frame}

\metroset{progressbar=none}
\begin{frame}[standout]
  \vfill
  \vfill
  \vfill
  \huge
  ¿Preguntas?
  \vfill
  \vfill
  \normalsize
  \textnormal{fersann1@upv.es}
\end{frame}

\end{document}
