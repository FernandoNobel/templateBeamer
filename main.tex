\documentclass[10pt,table]{beamer}

\usetheme{metropolis}

\metroset{progressbar=foot}
\metroset{titleformat=allsmallcaps}
\metroset{numbering=counter}

\graphicspath{{./img/}}
\titlegraphic{\vspace{60mm}\hfill\includegraphics[height=2cm]{logoUPV.pdf}}

\title{Minimal model for protein expression accounting for metabolic burden}
\subtitle{Automatización Industrial}
\author{Fernando Nóbel (\texttt{fersann1@upv.es})}
\date{Marzo 10, 2020}
\institute{Universitat Polit\`ecnica de Val\`encia}

\usepackage[framed,numbered,autolinebreaks,useliterate]{mcode} % Matlab code highlight

\usepackage{pgfplots}
\pgfplotsset{compat=newest}
%% the following commands are needed for some matlab2tikz features
\usetikzlibrary{plotmarks}
\usetikzlibrary{arrows.meta}
\usepgfplotslibrary{patchplots}
\usepackage{grffile}
\usepackage{amsmath}
\usepackage{tcolorbox}

\begin{document}


\maketitle

\begin{frame}{Table of contents}
  \setbeamertemplate{section in toc}[sections numbered]
  \tableofcontents[hideallsubsections]
\end{frame}

\section{Introducción}

\begin{frame}{Introducción}

  \begin{columns}[T,onlytextwidth]
    \column{0.49\textwidth}
    \metroset{block=fill}
    \begin{exampleblock}{Prácticas 1 y 2}
      \begin{enumerate}
        \item Diseño con Red de Petri.
        \item Uso de botonera para simular entradas.
      \end{enumerate}
    \end{exampleblock}

    \column{0.49\textwidth}
    \metroset{block=fill}
    \begin{alertblock}{Prácticas 3 y 4}
      \begin{enumerate}
        \item Diseño con GRAFCET.
        \item Uso del panel de cilindros neumáticos.
      \end{enumerate}
    \end{alertblock}

  \end{columns}
\end{frame}
\metroset{progressbar=none}

\begin{frame}[standout]
  ¿Preguntas?
\end{frame}

\end{document}
